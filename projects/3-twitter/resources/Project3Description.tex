\documentclass[11pt, oneside]{article}   	% use "amsart" instead of "article" for AMSLaTeX format
\usepackage{geometry}                		% See geometry.pdf to learn the layout options. There are lots.
\geometry{letterpaper}                   		% ... or a4paper or a5paper or ... 
%\geometry{landscape}                		% Activate for rotated page geometry
%\usepackage[parfill]{parskip}    		% Activate to begin paragraphs with an empty line rather than an indent
\usepackage{graphicx}				% Use pdf, png, jpg, or eps§ with pdflatex; use eps in DVI mode
								% TeX will automatically convert eps --> pdf in pdflatex		
\usepackage{amssymb}

%-----------------------------------------------------------------------------
% Special-purpose color definitions (dark enough to print OK in black and white)
\usepackage{color}
% A few colors to replace the defaults for certain link types
\definecolor{orange}{cmyk}{0,0.4,0.8,0.2}
\definecolor{darkorange}{rgb}{.71,0.21,0.01}
\definecolor{darkgreen}{rgb}{.12,.54,.11}
%-----------------------------------------------------------------------------
% The hyperref package gives us a pdf with properly built
% internal navigation ('pdf bookmarks' for the table of contents,
% internal cross-reference links, web links for URLs, etc.)
\usepackage{hyperref}
\hypersetup{pdftex, % needed for pdflatex
  breaklinks=true, % so long urls are correctly broken across lines
  colorlinks=true,
  urlcolor=blue,
  linkcolor=darkorange,
  citecolor=darkgreen,
}

\usepackage{booktabs}


\title{Stat 222 Project 3: Twitter}
\date{}							% Activate to display a given date or no date

\begin{document}
\maketitle

\section{Data Description}

For this project, your primary data source will be Twitter.  And you will be
expected to work with the Twitter API using Python to access this data. On
Monday (2/23), we will briefly review Python, introduce JSON (Javascript
Object Notation), and demonstrate how to interact with the Twitter API using
the \href{https://github.com/sixohsix/twitter}{Python Twitter Tools}.  However,
you will need to do outside reading to get up to speed with these
tools. 

Python Twitter Tools is one of several Python
packages\footnote{\url{http://www.danielforsyth.me/analyzing-a-nhl-playoff-game-with-twitter}}
for interacting with the Twitter API.  It is fairly minimal and is the package used in
chapter 1 and 9 of \emph{Mining the Social Web}.\footnote{\url{https://github.com/ptwobrussell/Mining-the-Social-Web-2nd-Edition}}
\begin{itemize}
\item \href{https://rawgit.com/ptwobrussell/Mining-the-Social-Web-2nd-Edition/master/ipynb/html/Chapter%201%20-%20Mining%20Twitter.html}{Chapter 1: Mining Twitter: Exploring Trending Topics, Discovering What People Are Talking About, and More}
\item \href{https://rawgit.com/ptwobrussell/Mining-the-Social-Web-2nd-Edition/master/ipynb/html/Chapter%209%20-%20Twitter%20Cookbook.html}{Chapter 9: Twitter Cookbook}
\item \href{http://nbviewer.ipython.org/github/ptwobrussell/Mining-the-Social-Web-2nd-Edition/tree/master/ipynb/}{Mining the Social Web notebooks}
\end{itemize}



\section{Your Assignment}

Each group is responsible for creating a presentation that visually answers a
set of questions, which you will determine for yourselves.  You will first need
to decide on a set of question that you can use Twitter data to answer.  Once
you determine the questions you wish to address, you will need to use the
Python twitter package to download the data from Twitter that you will use to
answer the question.  You are free to use Python to analyze the data and create
plots.  Since you've had limited practice with Python, you are welcome to use R
for some of the analysis and for creating your figures. However, even if you
decide to use R for the some of the analysis and plotting, you must use Python
to retrieve the data as well as most of the preprocessing.  Using Python save
the data as a CSV file, which you can then read with R.

{\em Which} questions you answer is up to you, but
think about telling a story. The story will be more interesting if the
questions you address are related to each other in some way.
Here are a few example topics:
\begin{itemize}
\item investigate the relation of breaking news on Twitter versus traditional
  news sources
\item compare stop words usage on Twitter versus NY Times
\item chart how the ratio of positive versus negative words used in tweets
  involving some event (or issue) change over time
\item relate tweets about a TV show/movie/book to their viewers/ticket sales/sales
  over time
\end{itemize}


\subsection*{Timeline}

Your final presentations are due in three weeks.  Here is the tentative schedule
of the next several classes:

\begin{table}[h]
\begin{tabular}{@{}l|l@{}}
\toprule
\multicolumn{1}{c|}{Monday} & \multicolumn{1}{c}{Wednesday}                 \\
\hline
(2/23) Start Twitter project & (2/25) Poster presentations for airline data \\
(3/2) Text mining            & (3/4) Pecha Kucha                            \\
(3/9) Group work             & (3/11) Practice presentations                \\
(3/16) Final presentations   &                                              \\                                       
\bottomrule
\end{tabular}
\end{table}

Note that you will have a practice presentation on the Wednesday, March 11th.
Your group will need to have already prepared and practiced your presentations
within your group prior to the 11th.  We expect a fairly polished talk for your
practice presentation.  After your practice presentation you will receive feedback
on how to improve your presentation.  Part of your final grade for this project
will involve how respond to feedback from your practice presentation in your
final presentation.

\section{Initial Guidance (to do for first data debrief)}

Since this is the first project for which you'll have to define and obtain the
data yourselves, the goal for the first week is for you \emph{to define what
tweets (or other info) you want to work with, download that data, and wrangle
it into a simpler format}. 

An important goal for this project is to provide an opportunity for you to
get more practice using Python.  In particular, for this project we expect
you to gain more experience working with basic Python structures (lists,
dictionaries, tuples, and strings) and work with JSON and CSV using Python.
You will also be using Python's string processing and text mining capabilities
to process the data.

We will reserve 20 minutes on Wednesday (2/25) for you to meet with your new
group and discuss possible topics/datasets. In preparation for that, you should
each come up with three ideas between Monday and Wednesday.  In addition to
the Twitter data, your group should also discuss what other data sources
you may need to use.

\section{Next Steps}

Next Monday (3/2), we will discuss text mining in Python using the Natural
Language Toolkit (nltk).\footnote{http://www.nltk.org} By this point, your
group should have downloaded the Twitter data and whatever other data you
think you will need.  

\section{Presentation Details}

All presentations will be given in the \textbf{Pecha
Kucha}\footnote{\url{http://en.wikipedia.org/wiki/PechaKucha}} style.  Pecha
Kucha presentations have a very strict format.  A Pecha Kucha presentation
consists of 20 slides that are automatically advanced 20 seconds (20x20).  The
complete presentation lasts exactly 6 minutes and 40 seconds.

This is a very constrained format.  So you will need to carefully plan and
prepare your talk.  Each group will have 4 members and each member will
be responsible for presenting 5 slides.  Since the slides will automatically
advance, you will need to practice your talks before you present in class.

Note that your group will need to have an official in-class practice presentation
on Wednesday (3/11).  After your practice presentation, you will receive feedback,
which you should incorporate in your final presentation on Monday (3/16).

\subsection*{How to make slides}

There are many ways to create slides, but make sure that you are able to
save your slides as a PDF.  Here are some possibilities for you to explore:
\begin{itemize}
\item Beamer\\
 \url{http://web.mit.edu/rsi/www/pdfs/beamer-tutorial.pdf}
\item Pandoc\\
 \url{http://johnmacfarlane.net/pandoc/demo/example9/producing-slide-shows-with-pandoc}
\item Powerpoint or Keynote
\end{itemize}

\subsection*{How \textbf{NOT} to make slides}

\begin{itemize}
\item Tufte's \emph{PowerPoint Is Evil}\\
 \url{http://archive.wired.com/wired/archive/11.09/ppt2.html}
\item Norvig's \emph{Gettysburg Cemetery Dedication}\\
 \url{http://norvig.com/Gettysburg/sld001.htm}
\item Efron's \emph{Thirteen rules}\\
 \url{http://statweb.stanford.edu/~ckirby/brad/other/2013ThirteenRules.pdf}
\end{itemize}

\end{document}  
