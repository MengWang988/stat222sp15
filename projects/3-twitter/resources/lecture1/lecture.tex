\section{Webservices}

Many data source are available on the Internet.  You've probably used
a web browser interface to search through some of this data and even
download it to your computer.  As you may have noticed, this manual
process is labor intensive, error prone, and hard to document.

To allow programmatic and automatic interaction with these data stores, many
website serve this information via a documented application programmer
interface (APIs).\footnote{See Appendix.}  These webservice APIs provide a
simple mechanism to create new functionality on top of existing webcontent.

While there are several ways to implement webservices,
\href{http://en.wikipedia.org/wiki/Representational_state_transfer}{Representational
State Transfer (REST)} has gained widespread popularity.  REST is more of a
style than a standard.  A system designed in the REST style is called RESTful.
RESTful systems typically use HTTP requests to read and post data using the
standard HTTP verbs (GET, POST, PUT, DELETE, etc.).  Often some form of
authentication is necessary to communicate with a webservice.  It is common to
use \href{http://en.wikipedia.org/wiki/OAuth}{OAuth} for this purpose.

\section{Data serialization}

Normally when you use a webbrowser to view a webpage, your browser handles
the HTTP communication for you.  You just specify what you wish to view
in the form of a \href{http://en.wikipedia.org/wiki/Uniform_resource_identifier}{URI}
such as \url{http://example.org/absolute/URI/with/absolute/path/to/resource.txt}.
At this point your browser communicates with the webserver and requests the
resource.  This resource is typically provided to your webbrowser as an
\href{http://en.wikipedia.org/wiki/HTML}{HTML} document, which webbrowsers know
how to render.

Similiarly, you will use HTTP to communicate with the Twitter webservice. However
rather than wishing to know how a webpage should look, you will be interested
in retrieving data in a form that is amenable to further processing.

Data serialization is the process of encoding data structures and objects in a
format that can be used to store this information on disk or transmit it over
the web.  For example, you may recall that in R you can use the Rdata format to
save R objects to disk and reload them later.  For webservices, JSON and XML
are standard formats.  To better understand this, let's briefly look at data
serialization more generally.


\subsection*{Python object}
First let's create a Python object.
<< d['src/serialize.py|idio|pycon|pyg|l']['mydict'] >>

And let's print the results:
<< d['src/serialize.py|idio|pycon|pyg|l']['pprint'] >>

\subsection*{XML}

How does this object look if we convert it to XML?\footnote{This functionality
is not part of the standard library.  And should not be used in practice.}

<< d['src/serialize.py|idio|pycon|pyg|l']['xml'] >>

\subsection*{JSON}
What if we convert it to JSON?
<< d['src/serialize.py|idio|pycon|pyg|l']['json'] >>

\subsection*{YAML}
What if we convert it to YAML?
<< d['src/serialize.py|idio|pycon|pyg|l']['yaml'] >>

\subsection*{Questions}
Looking over the output of the above formats you should notice several things.

\begin{itemize}
\item Which of the formats uses the largest number of characters?
\item Which uses the fewest?
\item Which looks most like Python?
\end{itemize}

\subsection*{Saving JSON and CSV}
For this project you will be querying the Twitter webservice and will
be getting responses in the JSON format.  After you get your response,
you will want to save it to disk.  I recommend that you use the JSON
format as your main data storage format for this project.

<< d['src/serialize.py|idio|pycon|pyg|l']['savejson'] >>

If you decide that you would like to use R for part of your analysis or for
creating figures, I recommend saving the information you want to work with in R
as a CSV file.  Your JSON file will have nested and non-homogeneous structure,
which is not possible to directly store using CSV.  So you will need to first
decide how to

% d['src/serialize.py|idio|pycon|pyg|l']['prepcsv'] >>

Now you can go ahead an save your list of tuples as a CSV file.

%    d['src/serialize.py|idio|pycon|pyg|l']['savecsv'] >>


\section{Example: US Senate tweets}


- connect to Twitter

- make some queries

- work with JSON

  - list

  - dictionary

  - string

  - list comprehension

- saving CSV

- work in R

  - pca / clustering?

- ex. senators

  - foreach senator

    - get name

    - get tweets

    - make term document

    - project labeled senators onto 1st and 2nd pcs

\newpage
\section*{Appendix}


\subsection*{Links}

\subsubsection*{Webservices} 
% add science resources
\begin{itemize}
\item \url{https://dev.twitter.com/overview/documentation}
\item \url{https://developers.facebook.com/docs/graph-api}
\item \url{https://developers.google.com/youtube/getting_started}
\item \url{http://en.wikipedia.org/w/api.php}
\item \url{http://www.mediawiki.org/wiki/API:Main_page}
\item \url{https://developer.github.com/v3/}
\end{itemize}

\subsubsection*{Serialization}

\begin{itemize}
\item \url{http://en.wikipedia.org/wiki/Serialization}
\item \url{http://en.wikipedia.org/wiki/Comparison_of_data_serialization_formats}
\item \url{http://www.json.org/xml.html}
\item \url{http://yaml.org/}
\item \url{http://www.drdobbs.com/web-development/after-xml-json-then-what/240151851}
\item \url{http://www.cowtowncoder.com/blog/archives/2012/04/entry_473.html}
\end{itemize}

